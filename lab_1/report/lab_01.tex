\documentclass[12pt]{report}
\usepackage[utf8]{inputenc}
\usepackage[russian]{babel}
%\usepackage[14pt]{extsizes}
\usepackage{listings}
\usepackage{graphicx}
\usepackage{amsmath,amsfonts,amssymb,amsthm,mathtools} 
\usepackage{float}

% Для листинга кода:
\lstset{ %
language=C++,                 % выбор языка для подсветки (здесь это С++)
basicstyle=\small\sffamily, % размер и начертание шрифта для подсветки кода
numbers=left,               % где поставить нумерацию строк (слева\справа)
numberstyle=\tiny,           % размер шрифта для номеров строк
stepnumber=1,                   % размер шага между двумя номерами строк
numbersep=7pt,                % как далеко отстоят номера строк от подсвечиваемого кода
showspaces=false,            % показывать или нет пробелы специальными отступами
showstringspaces=false,      % показывать или нет пробелы в строках
showtabs=false,             % показывать или нет табуляцию в строках
frame=single,              % рисовать рамку вокруг кода
tabsize=4,                 % размер табуляции по умолчанию равен 2 пробелам
captionpos=t,              % позиция заголовка вверху [t] или внизу [b] 
breaklines=true,           % автоматически переносить строки (да\нет)
breakatwhitespace=false, % переносить строки только если есть пробел
escapeinside={\#*}{*)}   % если нужно добавить комментарии в коде
}

% Для измененных титулов глав:
\usepackage{titlesec, blindtext, color} % подключаем нужные пакеты
\definecolor{gray75}{gray}{0.75} % определяем цвет
\newcommand{\hsp}{\hspace{20pt}} % длина линии в 20pt
% titleformat определяет стиль
\titleformat{\chapter}[hang]{\Huge\bfseries}{\thechapter\hsp\textcolor{gray75}{|}\hsp}{0pt}{\Huge\bfseries}


% plot
\usepackage{pgfplots}
\usepackage{filecontents}
\usetikzlibrary{datavisualization}
\usetikzlibrary{datavisualization.formats.functions}
\begin{filecontents}{LevR.dat}
1 1000
2 2114
3 6811
4 30522
5 115868
6 561283
7 6743323
\end{filecontents}

\begin{filecontents}{LevT.dat}
1 3818630
2 6861339
3 11611191
4 17771123
5 23581501
6 31673531
7 41037416
\end{filecontents}

\begin{filecontents}{DamLevR.dat}
1 1100
2 2420
3 7507
4 26990
5 120615
6 621520
7 7336250
\end{filecontents}

\begin{filecontents}{DamLevT.dat}
1 4452931
2 6394623
3 11380591
4 16096365
5 20575175
6 30105181
7 41091310
\end{filecontents}


\begin{filecontents}{LevRM.dat}
1 34
2 180
3 1006
4 5776
5 76711
6 197756
7 1167334
\end{filecontents}

\begin{filecontents}{LevTM.dat}
1 34
2 56
3 86
4 124
5 195
6 224
7 286
\end{filecontents}

\begin{filecontents}{DamLevRM.dat}
1 38
2 204
3 1158
4 6416
5 84015
6 215732
7 1264610
\end{filecontents}

\begin{filecontents}{DamLevTM.dat}
1 26
2 40
3 62
4 92
5 151
6 176
7 230
\end{filecontents}


\begin{document}
%\def\chaptername{} % убирает "Глава"
\begin{titlepage}
	\centering
	{\scshape\LARGE МГТУ им. Баумана \par}
	\vspace{3cm}
	{\scshape\Large Лабораторная работа №1\par}
	\vspace{0.5cm}	
	{\scshape\Large По курсу: "Анализ алгоритмов"\par}
	\vspace{1.5cm}
	{\huge\bfseries Расстояние Левенштейна\par}
	\vspace{2cm}
	\Large Работу выполнила: Серёгина Дарья, ИУ7-54Б\par
	\vspace{0.5cm}
	\Large Преподаватели:  Волкова Л.Л., Строганов Ю.В.\par

	\vfill
	\large \textit {Москва, 2020} \par
\end{titlepage}

\tableofcontents

\newpage
\chapter*{Введение}
\addcontentsline{toc}{chapter}{Введение}
\textbf{Расстояние Левенштейна} - согласно [4] - минимальное количество операций вставки одного символа, удаления одного символа и замены одного символа на другой, необходимых для превращения одной строки в другую.

Расстояние Левенштейна применяется в теории информации и компьютерной лингвистике для:

\begin{itemize}
	\item исправления ошибок в слове;
	\item сравнения текстовых файлов (утилита diff);
	\item в биоинформатике для сравнения генов, хромосом и белков.
\end{itemize}

Целью данной лабораторной работы является изучение метода динамического программирования на примере алгоритмов
Левенштейна и Дамерау-Левенштейна. 

Задачами лабораторной работы являются:
\begin{itemize}
  	\item изучение алгоритмов Левенштейна и Дамерау-Левенштейна нахождения расстояния между строками;
	\item реализация рекурсивной и динамической вариации указанных алгоритмов; 
	\item тестирование реализованных алгоритмов; 
	\item проведение сравнительного анализа алгоритмов по затрачиваемым ресурсам (времени и памяти).
\end{itemize}


\chapter{Аналитическая часть}
Задача по нахождению расстояния Левенштейна заключается в поиске минимального количества операций вставки/удаления/замены для превращения одной строки в другую.

При нахождении расстояния Дамерау — Левенштейна добавляется операция транспозиции (перестановки соседних символов). Полное определение рассмотрено в [1].  
 
\textbf{Действия обозначаются так:} 
\begin{itemize}
  	\item D (англ. delete) — удалить;
	\item I (англ. insert) — вставить;
	\item R (replace) — заменить;
	\item M(match) - совпадение.
\end{itemize}

Пусть $S_{1}$ и $S_{2}$ — две строки (длиной M и N соответственно) над некоторым алфавитом, тогда расстояние Левенштейна можно подсчитать по рекуррентной формуле (1.1), см [3]:


\begin{equation}
D(i,j) = \left\{ \begin{array}{ll}
 0, & \textrm{$i = 0, j = 0$}\\
 i, & \textrm{$j = 0, i > 0$}\\
 j, & \textrm{$i = 0, j > 0$}\\
min(\\
D(i,j-1)+1,\\
D(i-1, j) +1, &\textrm{$j>0, i>0$}\\
D(i-1, j-1) + m(S_{1}[i], S_{2}[j])\\
),
\end{array} \right.
\end{equation}

где $m(a,b)$ равна нулю, если $a=b$ и единице в противном случае; $min\{\,a,b,c\}$ возвращает наименьший из аргументов.

Расстояние Дамерау-Левенштейна вычисляется по рекуррентной формуле (1.2), см [2]:
		\begin{equation}
		     D(i, j) =  \left\{
			\begin{aligned}
				&0, && i = 0, j = 0\\
		    	&i, && i > 0, j = 0\\
		    	&j, && i = 0, j > 0\\		    	
		    	&min \left\{
				\begin{aligned}
					&D(i, j - 1) + 1,\\
		            &D(i - 1, j) + 1,\\
		            &D(i - 1, j - 1) + m(S_{1}[i], S_{2}[i]), \\
		            &D(i - 2, j - 2) + m(S_{1}[i], S_{2}[i]),\\
		        \end{aligned} \right.
		        && 
				\begin{aligned}
					&, \text{ если } i, j > 0 \\
		            & \text{ и } S_{1}[i] = S_{2}[j - 1] \\
		            & \text{ и } S_{1}[i - 1] =  S_{2}[j] \\
		        \end{aligned} \\ 
		        &min \left\{
		        \begin{aligned}
		            &D(i, j - 1) + 1,\\
		            &D(i - 1, j) + 1, \\
		            &D(i - 1, j - 1) + m(S_{1}[i], S_{2}[i]),\\
		        \end{aligned} \right.  &&, \text{иначе}
			\end{aligned} \right.
		\end{equation}
	    
		\section{Вывод}
		В данном разделе были рассмотрены алгоритмы нахождения расстояния Левенштейна и Дамерау-Левенштейна, который является модификаций первого, учитывающего возможность перестановки соседних символов. 
 



\chapter{Конструкторская часть}
\textbf{Ввод:}
\begin{itemize}
  	\item на вход подаются две строки;
  	\item строки могут быть пустыми, содержать пробелы, а также любые печатные символы UTF-8;
	\item uppercase и lowercase буквы считаются разными.
\end{itemize}
\textbf{Вывод:}
\begin{itemize}
  	\item программа выводит посчитанные каждым из алгоритмов расстояния;
  	\item для динамических реализаций алгоритмов выводятся заполненные матрицы;
  	\item в режиме замера ресурсов программа выводит средние время и память, затраченные каждым алгоритмом. 
\end{itemize}
\section{Схемы алгоритмов}
В данной части будут рассмотрены схемы алгоритмов Схемы рекурсивного алгоритма нахождения расстояния Левенштейна, матричного алгоритма нахождения расстояния Левенштейна, рекурсивного алгоритма нахождения расстояния Дамерау-Левенштейна и матричного алгоритма нахождения расстояния Дамерау-Левенштейна показаны на рисунках 2.1, 2.2, 2.3 и 2.4, соответственно.

\begin{figure}[h]
\centering
\includegraphics[width=0.7\linewidth]{RecLev.png}
\caption{Схема рекурсивного алгоритма нахождения расстояния Левенштейна}
\label{fig:mpr}
\end{figure}

\begin{figure}[h]
\centering
\includegraphics[width=0.63\linewidth]{MatrixL.jpg}
\caption{Схема матричного алгоритма нахождения расстояния Левенштейна}
\label{fig:mpr}
\end{figure}

\begin{figure}[h]
\centering
\includegraphics[width=0.7\linewidth]{RecDL.png}
\caption{Схема рекурсивного алгоритма нахождения расстояния Дамерау-Левенштейна}
\label{fig:mpr}
\end{figure}

\begin{figure}[h]
\centering
\includegraphics[width=0.8\linewidth]{MatrixDL.jpg}
\caption{Схема матричного алгоритма нахождения расстояния Дамерау-Левенштейна}
\label{fig:mpr}
\end{figure}


\chapter{Технологическая часть}
\section{Выбор ЯП}
Для реализации программы был выбран C++ из-за наличия опыта разработки на данном языке программирования. Среда разработки - Visual Studio Code.


\section{Реализация алгоритма}

\begin{lstlisting}[label=some-code,caption=Функция нахождения расстояния Левенштейна рекурсивно]
int levenstein_rec(string s1, string s2)
{
	int var = 1;
	int dist = 0;
	int s1_l, s2_l;
	
	s1_l = Llen(s1);
	s2_l = Llen(s2);
	
	
	if (s1.length() == 0 || s2.length() == 0) {
		dist = fabs(s1.length() - s2.length());
		return dist;
	}
	
	string s1_new = s1.substr(0, s1_l);
	string s2_new = s2.substr(0, s2_l);
	
	if (s1[s1_l] == s2[s2_l]) {
		var = 0;
	}
	
	return Mmin (levenstein_rec(s1_new, s2) + 1, levenstein_rec(s1, s2_new) + 1, levenstein_rec(s1_new, s2_new) + var);
}

\end{lstlisting}


\begin{lstlisting}[label=some-code,caption=Функция нахождения расстояния Левенштейна матрично]
int levenstein(string s1, string s2)
{
	int len_s1 = s1.length() + 1;
	int len_s2 = s2.length() + 1;
	
	int arr[len_s1][len_s2];
	
	for (int i = 0; i < len_s1; i++) {
		for (int j = 0; j < len_s2; j++) {
			if (i * j == 0) {
				arr[i][j] = i + j;
			} else {
				arr[i][j] = 0;
				
			}
		}
	}
	
	cout << s1 << endl;
	cout << s2 << endl;
	
	
	for (int i = 1; i < len_s1; i++) {
		for (int j = 1; j < len_s2; j++) {
			int key = 1;
			if (s1[i-1] == s2[j-1]) {
				key = 0;
			}
			arr[i][j] = Mmin(arr[i-1][j] + 1, arr[i][j-1] + 1,  arr[i-1][j-1] + key);
		}
	}
	
	for (int i = 0; i < len_s1; i++) {
		for (int j = 0; j < len_s2; j++) {
			cout << arr[i][j] << " ";
		}
		cout << "\n";
	}
	cout << "\n";
	
	return arr[len_s1 - 1][len_s2 - 1];
}

\end{lstlisting}


\begin{lstlisting}[label=some-code,caption=Функция нахождения расстояния Дамерау-Левенштейна рекурсивно]
int dameray_levenstein_rec(string s1, string s2)
{
	int var = 1;
	int dist = 0;
	int s1_l, s2_l;
	
	s1_l = Llen(s1);
	s2_l = Llen(s2);
	
	if (s1 == "" || s2 == "") {
		dist = max(s1.length(), s2.length());
		return dist;
	}
	
	string s1_new = s1.substr(0, s1.length() - 1);
	string s2_new = s2.substr(0, s2.length() - 1);
	
	if (s1[s1_l] == s2[s2_l]) {
		var = 0;
	}
	
	dist = Mmin (dameray_levenstein_rec(s1_new, s2) + 1,
	dameray_levenstein_rec(s1, s2_new) + 1,
	dameray_levenstein_rec(s1_new, s2_new) + var);
	
	if (s1.length() >= 2 && s2.length() >= 2 && s1[s1_l] == s2[s2_l - 1] &&
	s1[s1_l - 1] == s2[s2_l]) {
		string s1_damer = s1.substr(0, s1.length() - 2);
		string s2_damer = s2.substr(0, s2.length() - 2);
		dist = Mmin(dist, dist, dameray_levenstein_rec(s1_damer, s2_damer) + 1);
	}
	
	return dist;
}
\end{lstlisting}

\begin{lstlisting}[label=some-code,caption=Функция нахождения расстояния Дамерау-Левенштейна матрично]
def levenshtein_damerau_matrix(str1, str2, output):
	len_i = len(str1) + 1
	len_j = len(str2) + 1
	table = [[i + j for j in range(len_j)] for i in range(len_i)]

	for i in range(1, len_i):
		for j in range(1, len_j):
			forfeit = 0 if (str1[i - 1] == str2[j - 1]) else 1
			table[i][j] = min(table[i - 1][j] + 1,
			table[i][j - 1] + 1,
			table[i - 1][j - 1] + forfeit)
			if (i > 1 and j > 1) and str1[i-1] == str2[j-2] and str1[i-2] == str2[j-1]:
				table[i][j] = min(table[i][j], table[i-2][j-2] + 1)
	if (output):        
		OutputTable(table, str1, str2)
	return(table[-1][-1]), memory
\end{lstlisting}


\chapter{Исследовательская часть}

\section{Сравнительный анализ на основе замеров времени работы алгоритмов}

Был проведен замер времени работы каждого из алгоритмов. Результат замера показан в таблице 4.1.


\begin{table} [ht]
	\caption{Время работы алгоритмов (в тиках)}
\begin{tabular}{|c c c c c|} 
 	\hline
	len & Lev(R) & Lev(T) & DamLev(R) & DamLev(T) \\ [0.8ex] 
 	\hline\hline
 	1 & 1000 & 3818630 & 1100 & 4452931\\
 	\hline
 	2 & 2114 & 6861339 & 2420 & 6394623\\
 	\hline
	3 & 6811 & 11611191 & 7507 & 11380591\\
	\hline
	4 & 30522 & 17771123 & 26990 & 16096365\\
	\hline
	5 & 115868 & 23581501 & 120615 & 20575176\\
	\hline
	6 & 561283 & 31673531 & 621520 & 30105181\\
	\hline
	7 & 6743323 & 41037416 & 7336250 & 41091310\\
	\hline
	\end{tabular}
\end{table}

Полученная зависимость времени работы алгоритмов от длины строк показана на рисунке 4.1.


\begin{figure} [H]
\begin{tikzpicture}
\begin{axis}[
    	axis lines = left,
    	xlabel={len (symbols)},
    	ylabel={time (ticks)},
    	xmin=1, xmax=7,
    	ymin=1, ymax=55000000,
	legend pos=north west,
	ymajorgrids=true
]
\addplot[color=red] table[x index=0, y index=1] {LevR.dat}; 
\addplot[color=orange] table[x index=0, y index=1] {DamLevR.dat};
\addplot[color=blue, mark=square] table[x index=0, y index=1] {LevT.dat};
\addplot[color=green, mark=square] table[x index=0, y index=1] {DamLevT.dat};

\addlegendentry{LevR}
\addlegendentry{DamLevR}
\addlegendentry{LevT}
\addlegendentry{DamLevT}
\end{axis}
\end{tikzpicture}
\caption{Зависимость времени работы от длины строк}
\end{figure}


\par
На основе проведённых измерений можно сделать вывод, что рекурсивные алгоритмы эффективней для коротких строк. Однако при увелечении длины, динамические алгоритмы выступают более эффективными, что обусловлено большим количеством повторных рассчетов в рекурсивных реализациях, в то время как в динамических реализациях ячейка матрицы расчитывается единожды. Также установлено, что алгоритм ДамерауЛевенштейна в среднем работает несколько дольше алгоритма Левенштейна, что объясняется наличием дополнительных проверок, однако алгоритмы сравнимы по временной эффективности.



\section{Сравнительный анализ алгоритмов на основе замеров затрачиваемой памяти}
Был проведен замер памяти, затрачиваемой алгоритмами. Результат замера показан в таблице 4.2.


\begin{table} [H]
	\caption{Количество памяти, затрачиваемой алгоритмом (в байтах)}
	\begin{tabular}{|c c c c c|} 
		\hline
		len & Lev(R) & Lev(T) & DamLev(R) & DamLev(T) \\ [0.8ex] 
		\hline\hline
		1 & 34 & 34 & 38 & 26\\
		\hline
		2 & 180 & 56 & 204 & 40\\
		\hline
		3 & 1006 & 86 & 1158 & 62\\
		\hline
		4 & 5776 & 124 & 6416 & 92\\
		\hline
		5 & 76711 & 195 & 84015 & 151\\
		\hline
		6 & 197756 & 224 & 215732 & 176\\
		\hline
		7 & 1167334 & 286 & 1264610 & 230\\
		\hline
	\end{tabular}
	
\end{table}



Полученная зависимость памяти затрачиваемой алгоритмами от длины строк показана на рисунке 4.2.

\begin{figure} [H]
	\begin{tikzpicture}
		\begin{axis}[
			axis lines = left,
			xlabel={len (symbols)},
			ylabel={memory (bytes)},
			xmin=1, xmax=7,
			ymin=1, ymax=1264610,
			legend pos=north west,
			ymajorgrids=true
			]
			\addplot[color=red] table[x index=0, y index=1] {LevRM.dat}; 
			\addplot[color=orange] table[x index=0, y index=1] {DamLevRM.dat};
			\addplot[color=blue, mark=square] table[x index=0, y index=1] {LevTM.dat};
			\addplot[color=green, mark=square] table[x index=0, y index=1] {DamLevTM.dat};
			
			\addlegendentry{LevR}
			\addlegendentry{DamLevR}
			\addlegendentry{LevT}
			\addlegendentry{DamLevT}
		\end{axis}
	\end{tikzpicture}
	\caption{Зависимость затрачиваемой памяти от длины строк}
\end{figure}

\par
На основе проведённых измерений можно сделать вывод, что рекурсивные алгоритмы сравнимы по количеству затрачиваемой памяти с динамическими при малых длинах входных строк. Однако при росте длины строк количество памяти, затрачиваемой рекурсивными алгоритмами резко возрастает из-за локальных переменных, создаваемых при каждом вызове алгоритма, в то время как память динамических алгоритмов изменяется слабо - только из-за увеличения хранимой матрицы.




\section{Тестовые данные}

\par
Проведем тестирование программы. В столбцах "Ожидаемый результат" и "Полученный результат" 4 числа соответсвуют рекурсивному алгоритму нахождения расстояния Левенштейна, матричному алгоритму нахождению расстоянию Левенштейна, рекурсивному алгоритму расстояния Дамерау-Левенштейна, матричному алгоритму нахождения расстояние Дамерау-Левенштейна.

\begin{table} [H]
\caption{Таблица тестовых данных}
	\begin{tabular}{|c c c c c|} 
 	\hline
	№ & Первое слово & Второе слово & Ожидаемый результат & Полученный результат \\ [0.8ex] 
 	\hline\hline
 	1 &  &  & 0 0 0 0 & 0 0 0 0\\
 	\hline
 	2 & kot & skat & 2 2 2 2 & 2 2 2 2\\
 	\hline
	3 & kate & ktae & 2 2 1 1 & 2 2 1 1\\
	\hline
	4 & abacaba & aabcaab & 4 4 2 2 & 4 4 2 2\\
	\hline
	5 & sobaka & sboku & 3 3 3 3 & 3 3 3 3\\
	\hline
	6 & qwerty & queue & 4 4 4 4 & 4 4 4 4\\
	\hline
	7 & apple & aplpe & 2 2 1 1  & 2 2 1 1\\
	\hline
	8 &  & cat & 3 3 3 3 & 3 3 3 3\\
	\hline
	9 & parallels &  & 9 9 9 9 & 9 9 9 9\\
	\hline
	10 & bmstu & utsmb & 4 4 4 4 & 4 4 4 4\\
	\hline
	\end{tabular}
\end{table}



\chapter*{Заключение}
\addcontentsline{toc}{chapter}{Заключение}
Были изучены методы динамического и рекурсивного программирования на примере алгоритмов Левенштейна и Дамерау-Левенштейна.
Получены практические навыки реализации указанных алгоритмов в матричной и динамической реализации.

Экспериментально было подтверждено различие во временной эффективности рекурсивной и нерекурсивной реализаций выбранного алгоритма определения расстояния между строками при помощи разработаного программного обеспечения на материале замеров процессорного времени выполнения реализации на варьирующихся длинах строк. 

В результате исследований можно прийти к вводу, что матричная реализация данных алгоритмов заметно выигрывает по времени при росте длины строк, следовательно более применима в реальных проектах.


\chapter*{Список использованных источников}
\begin{enumerate}
	\item Задача о расстоянии Дамерау-Левенштейна [Электронный ресурс]. – Режим доступа: https://neerc.ifmo.ru/wiki/index.php?title=Задача-о-расстоянии-Дамерау-Левенштейна. – Дата доступа: 21.09.2020.
	\item Вычисление редакционного расстояния [Электронный ресурс]. – Режим доступа: https://habr.com/ru/post/117063/. – Дата доступа: 21.09.2020.
	\item Вычисление расстояния Левенштейна [Электронный ресурс]. – Режим доступа: https://foxford.ru/wiki/informatika/vychislenie-rasstoyaniya-levenshteyna. – Дата доступа: 21.09.2020.
	\item Расстояние Левенштейна [Электронный ресурс]. – Режим доступа: https://vc.ru/newtechaudit/129654-rasstoyanie-levenshteyna-dlya-poiska-opechatok-v-dannyh-klienta. – Дата доступа: 21.09.2020.
\end{enumerate}
\end{document}